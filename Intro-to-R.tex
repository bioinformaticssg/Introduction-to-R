\documentclass[]{article}
\usepackage{lmodern}
\usepackage{amssymb,amsmath}
\usepackage{ifxetex,ifluatex}
\usepackage{fixltx2e} % provides \textsubscript
\ifnum 0\ifxetex 1\fi\ifluatex 1\fi=0 % if pdftex
  \usepackage[T1]{fontenc}
  \usepackage[utf8]{inputenc}
\else % if luatex or xelatex
  \ifxetex
    \usepackage{mathspec}
  \else
    \usepackage{fontspec}
  \fi
  \defaultfontfeatures{Ligatures=TeX,Scale=MatchLowercase}
\fi
% use upquote if available, for straight quotes in verbatim environments
\IfFileExists{upquote.sty}{\usepackage{upquote}}{}
% use microtype if available
\IfFileExists{microtype.sty}{%
\usepackage{microtype}
\UseMicrotypeSet[protrusion]{basicmath} % disable protrusion for tt fonts
}{}
\usepackage[margin=1in]{geometry}
\usepackage{hyperref}
\hypersetup{unicode=true,
            pdftitle={Introduction to R},
            pdfauthor={Stacey Borrego},
            pdfborder={0 0 0},
            breaklinks=true}
\urlstyle{same}  % don't use monospace font for urls
\usepackage{color}
\usepackage{fancyvrb}
\newcommand{\VerbBar}{|}
\newcommand{\VERB}{\Verb[commandchars=\\\{\}]}
\DefineVerbatimEnvironment{Highlighting}{Verbatim}{commandchars=\\\{\}}
% Add ',fontsize=\small' for more characters per line
\usepackage{framed}
\definecolor{shadecolor}{RGB}{248,248,248}
\newenvironment{Shaded}{\begin{snugshade}}{\end{snugshade}}
\newcommand{\KeywordTok}[1]{\textcolor[rgb]{0.13,0.29,0.53}{\textbf{#1}}}
\newcommand{\DataTypeTok}[1]{\textcolor[rgb]{0.13,0.29,0.53}{#1}}
\newcommand{\DecValTok}[1]{\textcolor[rgb]{0.00,0.00,0.81}{#1}}
\newcommand{\BaseNTok}[1]{\textcolor[rgb]{0.00,0.00,0.81}{#1}}
\newcommand{\FloatTok}[1]{\textcolor[rgb]{0.00,0.00,0.81}{#1}}
\newcommand{\ConstantTok}[1]{\textcolor[rgb]{0.00,0.00,0.00}{#1}}
\newcommand{\CharTok}[1]{\textcolor[rgb]{0.31,0.60,0.02}{#1}}
\newcommand{\SpecialCharTok}[1]{\textcolor[rgb]{0.00,0.00,0.00}{#1}}
\newcommand{\StringTok}[1]{\textcolor[rgb]{0.31,0.60,0.02}{#1}}
\newcommand{\VerbatimStringTok}[1]{\textcolor[rgb]{0.31,0.60,0.02}{#1}}
\newcommand{\SpecialStringTok}[1]{\textcolor[rgb]{0.31,0.60,0.02}{#1}}
\newcommand{\ImportTok}[1]{#1}
\newcommand{\CommentTok}[1]{\textcolor[rgb]{0.56,0.35,0.01}{\textit{#1}}}
\newcommand{\DocumentationTok}[1]{\textcolor[rgb]{0.56,0.35,0.01}{\textbf{\textit{#1}}}}
\newcommand{\AnnotationTok}[1]{\textcolor[rgb]{0.56,0.35,0.01}{\textbf{\textit{#1}}}}
\newcommand{\CommentVarTok}[1]{\textcolor[rgb]{0.56,0.35,0.01}{\textbf{\textit{#1}}}}
\newcommand{\OtherTok}[1]{\textcolor[rgb]{0.56,0.35,0.01}{#1}}
\newcommand{\FunctionTok}[1]{\textcolor[rgb]{0.00,0.00,0.00}{#1}}
\newcommand{\VariableTok}[1]{\textcolor[rgb]{0.00,0.00,0.00}{#1}}
\newcommand{\ControlFlowTok}[1]{\textcolor[rgb]{0.13,0.29,0.53}{\textbf{#1}}}
\newcommand{\OperatorTok}[1]{\textcolor[rgb]{0.81,0.36,0.00}{\textbf{#1}}}
\newcommand{\BuiltInTok}[1]{#1}
\newcommand{\ExtensionTok}[1]{#1}
\newcommand{\PreprocessorTok}[1]{\textcolor[rgb]{0.56,0.35,0.01}{\textit{#1}}}
\newcommand{\AttributeTok}[1]{\textcolor[rgb]{0.77,0.63,0.00}{#1}}
\newcommand{\RegionMarkerTok}[1]{#1}
\newcommand{\InformationTok}[1]{\textcolor[rgb]{0.56,0.35,0.01}{\textbf{\textit{#1}}}}
\newcommand{\WarningTok}[1]{\textcolor[rgb]{0.56,0.35,0.01}{\textbf{\textit{#1}}}}
\newcommand{\AlertTok}[1]{\textcolor[rgb]{0.94,0.16,0.16}{#1}}
\newcommand{\ErrorTok}[1]{\textcolor[rgb]{0.64,0.00,0.00}{\textbf{#1}}}
\newcommand{\NormalTok}[1]{#1}
\usepackage{graphicx,grffile}
\makeatletter
\def\maxwidth{\ifdim\Gin@nat@width>\linewidth\linewidth\else\Gin@nat@width\fi}
\def\maxheight{\ifdim\Gin@nat@height>\textheight\textheight\else\Gin@nat@height\fi}
\makeatother
% Scale images if necessary, so that they will not overflow the page
% margins by default, and it is still possible to overwrite the defaults
% using explicit options in \includegraphics[width, height, ...]{}
\setkeys{Gin}{width=\maxwidth,height=\maxheight,keepaspectratio}
\IfFileExists{parskip.sty}{%
\usepackage{parskip}
}{% else
\setlength{\parindent}{0pt}
\setlength{\parskip}{6pt plus 2pt minus 1pt}
}
\setlength{\emergencystretch}{3em}  % prevent overfull lines
\providecommand{\tightlist}{%
  \setlength{\itemsep}{0pt}\setlength{\parskip}{0pt}}
\setcounter{secnumdepth}{0}
% Redefines (sub)paragraphs to behave more like sections
\ifx\paragraph\undefined\else
\let\oldparagraph\paragraph
\renewcommand{\paragraph}[1]{\oldparagraph{#1}\mbox{}}
\fi
\ifx\subparagraph\undefined\else
\let\oldsubparagraph\subparagraph
\renewcommand{\subparagraph}[1]{\oldsubparagraph{#1}\mbox{}}
\fi

%%% Use protect on footnotes to avoid problems with footnotes in titles
\let\rmarkdownfootnote\footnote%
\def\footnote{\protect\rmarkdownfootnote}

%%% Change title format to be more compact
\usepackage{titling}

% Create subtitle command for use in maketitle
\newcommand{\subtitle}[1]{
  \posttitle{
    \begin{center}\large#1\end{center}
    }
}

\setlength{\droptitle}{-2em}
  \title{Introduction to R}
  \pretitle{\vspace{\droptitle}\centering\huge}
  \posttitle{\par}
  \author{Stacey Borrego}
  \preauthor{\centering\large\emph}
  \postauthor{\par}
  \predate{\centering\large\emph}
  \postdate{\par}
  \date{4/07/2018}


\begin{document}
\maketitle

\subsubsection{Purpose}\label{purpose}

The purpose of this document is to give a very brief and digestable
introduction to R. There are so many details in every programming
language and certainly more thorough explanations are needed than I will
provide here. My goal with this document is to expose you to R and
hopefully pique your interest to learn more.

\subsubsection{Resources used for this
document}\label{resources-used-for-this-document}

\begin{itemize}
\tightlist
\item
  \href{http://shop.oreilly.com/product/0636920030157.do}{\emph{Bioinformatics
  Data Skills: Reproducible and Robust Research with Open Source Tools}}
  by Vince Buffalo (O'Reilly 2015)
\item
  \href{https://www.rstudio.com/wp-content/uploads/2015/02/rmarkdown-cheatsheet.pdf}{R
  Markdown Cheat Sheet}
\end{itemize}

\subsubsection{Some Notes about RStudio and
R}\label{some-notes-about-rstudio-and-r}

\paragraph{RStudio}\label{rstudio}

\begin{itemize}
\tightlist
\item
  RStudio provides an enviroment with helpful tools for you to use and
  build scripts in R. Your basic enviroment will have at least your
  Console and Source (basically a text editor for you to write your
  script/file). You can change the layout of your RStudio enviroment by
  going to Tools \textgreater{} Global Options \textgreater{} Pane
  Layout.
\item
  Learn more about
  \href{https://www.rstudio.com/products/rstudio/features/}{RStudio
  features}.
\end{itemize}

\paragraph{R}\label{r}

\begin{itemize}
\tightlist
\item
  Some programming languages are very picky about spacing, but R is
  usually not. That being said, there are style conventions that are
  recommended. These conventions extend from spacing to naming and
  everything in between. Here is a
  \href{http://adv-r.had.co.nz/Style.html}{Style Guide} to get you
  started.
\end{itemize}

\section{R Basics}\label{r-basics}

\subsection{Simple Math}\label{simple-math}

Let's get started by writing some simple math expressions for R to
evaluate.

You can run expressions in one of two ways, or both if you are curious!

\begin{enumerate}
\def\labelenumi{\arabic{enumi}.}
\tightlist
\item
  Copy or type each expression into the RStudio Console. Hit
  Enter/Return.
\item
  Open a new R script (File \textgreater{} New File \textgreater{} R
  script).

  \begin{itemize}
  \tightlist
  \item
    Copy the expression into the new file.
  \item
    Place your cursor on the line you want to run, or select everything.
  \item
    Hit the run button on the top right. This will run the expression on
    the current line your cursor is on or whatever is selected.
  \item
    Keyboard Shortcut to run your script/expression: Command-Enter (OS
    X) or Control-Enter (Windows, Linux)
  \end{itemize}
\end{enumerate}

\begin{Shaded}
\begin{Highlighting}[]
\DecValTok{4} \OperatorTok{+}\StringTok{ }\DecValTok{3}
\end{Highlighting}
\end{Shaded}

\begin{verbatim}
## [1] 7
\end{verbatim}

\begin{Shaded}
\begin{Highlighting}[]
\DecValTok{4} \OperatorTok{-}\StringTok{ }\DecValTok{3}
\end{Highlighting}
\end{Shaded}

\begin{verbatim}
## [1] 1
\end{verbatim}

\begin{Shaded}
\begin{Highlighting}[]
\DecValTok{4} \OperatorTok{*}\StringTok{ }\DecValTok{3}
\end{Highlighting}
\end{Shaded}

\begin{verbatim}
## [1] 12
\end{verbatim}

\begin{Shaded}
\begin{Highlighting}[]
\DecValTok{4} \OperatorTok{/}\StringTok{ }\DecValTok{3}
\end{Highlighting}
\end{Shaded}

\begin{verbatim}
## [1] 1.333333
\end{verbatim}

To indicate the order of evaluation, use parentheses to indicate which
expression should be evaluated first.

\begin{Shaded}
\begin{Highlighting}[]
\DecValTok{4} \OperatorTok{+}\StringTok{ }\DecValTok{3}\OperatorTok{/}\DecValTok{2}
\end{Highlighting}
\end{Shaded}

\begin{verbatim}
## [1] 5.5
\end{verbatim}

\begin{Shaded}
\begin{Highlighting}[]
\NormalTok{(}\DecValTok{4} \OperatorTok{+}\StringTok{ }\DecValTok{3}\NormalTok{)}\OperatorTok{/}\DecValTok{2}
\end{Highlighting}
\end{Shaded}

\begin{verbatim}
## [1] 3.5
\end{verbatim}

We can also use functions to perform mathematical operations. Functions
are written with the function name followed by parentheses, no spaces! A
function takes zero or more arguments, evaluates the input, and outputs
a return value.

To find the square root of a number, we can use the function sqrt().

\begin{Shaded}
\begin{Highlighting}[]
\KeywordTok{sqrt}\NormalTok{(}\DecValTok{4}\NormalTok{)}
\end{Highlighting}
\end{Shaded}

\begin{verbatim}
## [1] 2
\end{verbatim}

\begin{Shaded}
\begin{Highlighting}[]
\KeywordTok{sqrt}\NormalTok{(}\DecValTok{2}\OperatorTok{*}\DecValTok{2}\NormalTok{)}
\end{Highlighting}
\end{Shaded}

\begin{verbatim}
## [1] 2
\end{verbatim}

\subsection{Getting Help}\label{getting-help}

To learn more about the arguments for a function you can use the
function args(), just provide the function name as the agrument.

\begin{Shaded}
\begin{Highlighting}[]
\KeywordTok{args}\NormalTok{(args)}
\end{Highlighting}
\end{Shaded}

\begin{verbatim}
## function (name) 
## NULL
\end{verbatim}

\begin{Shaded}
\begin{Highlighting}[]
\KeywordTok{args}\NormalTok{(sqrt)}
\end{Highlighting}
\end{Shaded}

\begin{verbatim}
## function (x) 
## NULL
\end{verbatim}

\begin{Shaded}
\begin{Highlighting}[]
\KeywordTok{args}\NormalTok{(plot)}
\end{Highlighting}
\end{Shaded}

\begin{verbatim}
## function (x, y, ...) 
## NULL
\end{verbatim}

If you want to read the documentation for a function, you can pull up a
special help window in Rstudio.

\begin{Shaded}
\begin{Highlighting}[]
\KeywordTok{help}\NormalTok{(sqrt)}
\NormalTok{?sqrt}
\end{Highlighting}
\end{Shaded}

\subsection{Variables}\label{variables}

Variables are a great way to save a value for future use. We can assign
a value to a symbol using the \textless{}- assignment operator.

\begin{Shaded}
\begin{Highlighting}[]
\NormalTok{x <-}\StringTok{ }\DecValTok{4}
\end{Highlighting}
\end{Shaded}

When we want to see what value our variable is assigned we can just type
in the variable symbol into the console. You can also look at the
Environment window in RStudio and it will show you all the variables you
have assigned and their values.

\begin{Shaded}
\begin{Highlighting}[]
\NormalTok{x}
\end{Highlighting}
\end{Shaded}

\begin{verbatim}
## [1] 4
\end{verbatim}

Once our variable has a value assigned, we can use it in functions.

\begin{Shaded}
\begin{Highlighting}[]
\KeywordTok{sqrt}\NormalTok{(x)}
\end{Highlighting}
\end{Shaded}

\begin{verbatim}
## [1] 2
\end{verbatim}

Variables can also be assigned the ouput of an expression.

\begin{Shaded}
\begin{Highlighting}[]
\NormalTok{square_result <-}\StringTok{ }\KeywordTok{sqrt}\NormalTok{(x)}
\NormalTok{square_result}
\end{Highlighting}
\end{Shaded}

\begin{verbatim}
## [1] 2
\end{verbatim}

\subsection{Vectors}\label{vectors}

R is most known for its use of vectors and vectorization. Everything is
stored in a vector; a single value is stored in a vector of 1. You can
see the length of a vector using the function length().

\begin{Shaded}
\begin{Highlighting}[]
\KeywordTok{length}\NormalTok{(x)}
\end{Highlighting}
\end{Shaded}

\begin{verbatim}
## [1] 1
\end{verbatim}

We can create longer vectors using c(), which stands for concatenate.

\begin{Shaded}
\begin{Highlighting}[]
\NormalTok{y <-}\StringTok{ }\KeywordTok{c}\NormalTok{(}\DecValTok{1}\NormalTok{, }\DecValTok{5}\NormalTok{, }\DecValTok{7}\NormalTok{)}
\NormalTok{y}
\end{Highlighting}
\end{Shaded}

\begin{verbatim}
## [1] 1 5 7
\end{verbatim}

\begin{Shaded}
\begin{Highlighting}[]
\KeywordTok{length}\NormalTok{(y)}
\end{Highlighting}
\end{Shaded}

\begin{verbatim}
## [1] 3
\end{verbatim}

Vectorization is the process by which a process is applied to a whole
array instead of a single element. This means if the variable \textbf{y}
were multiplied by 2, each element will be muliplied by 2 but the vector
length will remain at 3.

\begin{Shaded}
\begin{Highlighting}[]
\NormalTok{y }\OperatorTok{*}\StringTok{ }\DecValTok{2}
\end{Highlighting}
\end{Shaded}

\begin{verbatim}
## [1]  2 10 14
\end{verbatim}

\begin{Shaded}
\begin{Highlighting}[]
\KeywordTok{length}\NormalTok{(y }\OperatorTok{*}\StringTok{ }\DecValTok{2}\NormalTok{)}
\end{Highlighting}
\end{Shaded}

\begin{verbatim}
## [1] 3
\end{verbatim}

Vectorization allows us to perform arithmetic operations on two vectors,
each operation occurring elementwise (eg a1 + b1, a2 + b2, a3 + b3
\ldots{})

\begin{Shaded}
\begin{Highlighting}[]
\NormalTok{a <-}\StringTok{ }\DecValTok{1}\OperatorTok{:}\DecValTok{3}
\NormalTok{a}
\end{Highlighting}
\end{Shaded}

\begin{verbatim}
## [1] 1 2 3
\end{verbatim}

\begin{Shaded}
\begin{Highlighting}[]
\NormalTok{b <-}\StringTok{ }\DecValTok{4}\OperatorTok{:}\DecValTok{6}
\NormalTok{b}
\end{Highlighting}
\end{Shaded}

\begin{verbatim}
## [1] 4 5 6
\end{verbatim}

\begin{Shaded}
\begin{Highlighting}[]
\NormalTok{a }\OperatorTok{+}\StringTok{ }\NormalTok{b}
\end{Highlighting}
\end{Shaded}

\begin{verbatim}
## [1] 5 7 9
\end{verbatim}

If the two vectors are not the same length, R will recycle the values of
the shorter vector (eg c1 + d1, c2 + d2, c3 + d1, c4 + d2)

\begin{Shaded}
\begin{Highlighting}[]
\NormalTok{c <-}\StringTok{ }\DecValTok{1}\OperatorTok{:}\DecValTok{4}
\NormalTok{c}
\end{Highlighting}
\end{Shaded}

\begin{verbatim}
## [1] 1 2 3 4
\end{verbatim}

\begin{Shaded}
\begin{Highlighting}[]
\NormalTok{d <-}\StringTok{ }\DecValTok{1}\OperatorTok{:}\DecValTok{2}
\NormalTok{d}
\end{Highlighting}
\end{Shaded}

\begin{verbatim}
## [1] 1 2
\end{verbatim}

\begin{Shaded}
\begin{Highlighting}[]
\NormalTok{c }\OperatorTok{+}\StringTok{ }\NormalTok{d}
\end{Highlighting}
\end{Shaded}

\begin{verbatim}
## [1] 2 4 4 6
\end{verbatim}

A warning will pop up if the longer vector length is not a multiple of
the shorter vector length. The operation will still be performed but R
thought it should say something.

\begin{Shaded}
\begin{Highlighting}[]
\NormalTok{e <-}\StringTok{ }\DecValTok{1}\OperatorTok{:}\DecValTok{5}
\NormalTok{e}
\end{Highlighting}
\end{Shaded}

\begin{verbatim}
## [1] 1 2 3 4 5
\end{verbatim}

\begin{Shaded}
\begin{Highlighting}[]
\NormalTok{f <-}\StringTok{ }\DecValTok{1}\OperatorTok{:}\DecValTok{2}
\NormalTok{f}
\end{Highlighting}
\end{Shaded}

\begin{verbatim}
## [1] 1 2
\end{verbatim}

\begin{Shaded}
\begin{Highlighting}[]
\NormalTok{e }\OperatorTok{+}\StringTok{ }\NormalTok{f}
\end{Highlighting}
\end{Shaded}

\begin{verbatim}
## Warning in e + f: longer object length is not a multiple of shorter object
## length
\end{verbatim}

\begin{verbatim}
## [1] 2 4 4 6 6
\end{verbatim}

We can also provide vectors as an argument to a function. Depending on
what the function does, your output will look differently. For example,
sqrt() only takes the square root of one number at a time, so it returns
a value for each number in the vector \textbf{y} -- sqrt(1), sqrt(5),
sqrt(7). However, mean() computes the average of a vector of numbers and
thus returns a single value.

\begin{Shaded}
\begin{Highlighting}[]
\NormalTok{y}
\end{Highlighting}
\end{Shaded}

\begin{verbatim}
## [1] 1 5 7
\end{verbatim}

\begin{Shaded}
\begin{Highlighting}[]
\KeywordTok{sqrt}\NormalTok{(y)}
\end{Highlighting}
\end{Shaded}

\begin{verbatim}
## [1] 1.000000 2.236068 2.645751
\end{verbatim}

\begin{Shaded}
\begin{Highlighting}[]
\KeywordTok{mean}\NormalTok{(y)}
\end{Highlighting}
\end{Shaded}

\begin{verbatim}
## [1] 4.333333
\end{verbatim}

Note: When you look up the arguments for a function, you will often see
\textbf{x} given as the first argument. The documentation will provide
more information as to what \textbf{x} is. For example when you look up
the documentation for square root (?sqrt), \textbf{x} is defined as ``a
numeric or complex array''. When you look at the documentation for mean
(?mean), \textbf{x} is defined as ``typically a vector-like object''.
Just don't confuse this with a variable that you may have set, it is
just a placeholder.

\subsection{Vector Types}\label{vector-types}

Vectors in R must contain elements of the same type. We can check the
vector type by using the function class() or typeof(). It is important
to note that R will coerce vectors of different types to the type that
leads to no information loss. For example, a vector containing both
characters and numeric data will be coerced into character values.

\textbf{Numeric (aka double)}

\begin{itemize}
\tightlist
\item
  Any real number
\item
  example: 4, -4, 4.4, -4.04
\item
  \emph{Test function}: is.numeric()

  \begin{itemize}
  \tightlist
  \item
    returns TRUE or FALSE value
  \end{itemize}
\item
  \emph{Coercion function}: as.numeric()

  \begin{itemize}
  \tightlist
  \item
    coerces values to be numeric
  \end{itemize}
\end{itemize}

\begin{Shaded}
\begin{Highlighting}[]
\NormalTok{num_example <-}\StringTok{ }\KeywordTok{c}\NormalTok{(}\DecValTok{4}\NormalTok{, }\OperatorTok{-}\DecValTok{4}\NormalTok{, }\FloatTok{4.4}\NormalTok{, }\OperatorTok{-}\FloatTok{4.04}\NormalTok{)}
\KeywordTok{class}\NormalTok{(num_example)}
\end{Highlighting}
\end{Shaded}

\begin{verbatim}
## [1] "numeric"
\end{verbatim}

\textbf{Integer}

\begin{itemize}
\tightlist
\item
  Any whole number
\item
  example: 4, -4, 44
\item
  By default, integer values are assigned as numeric. It must be
  explicitly indicated that the values should be treated as integers.
\item
  \emph{Test function}: is.integer()
\item
  \emph{Coercion function}: as.integer()
\item
  Will sometimes be indicated with an L after the number.
\end{itemize}

\begin{Shaded}
\begin{Highlighting}[]
\NormalTok{int_example <-}\StringTok{ }\KeywordTok{c}\NormalTok{(}\DecValTok{4}\NormalTok{, }\OperatorTok{-}\DecValTok{4}\NormalTok{, }\DecValTok{44}\NormalTok{)}
\KeywordTok{class}\NormalTok{(int_example)}
\end{Highlighting}
\end{Shaded}

\begin{verbatim}
## [1] "numeric"
\end{verbatim}

\begin{Shaded}
\begin{Highlighting}[]
\NormalTok{int_example <-}\StringTok{ }\KeywordTok{as.integer}\NormalTok{(int_example)}
\KeywordTok{class}\NormalTok{(int_example)}
\end{Highlighting}
\end{Shaded}

\begin{verbatim}
## [1] "integer"
\end{verbatim}

\begin{Shaded}
\begin{Highlighting}[]
\NormalTok{int_example2 <-}\StringTok{ }\DecValTok{5}
\KeywordTok{is.integer}\NormalTok{(int_example2)}
\end{Highlighting}
\end{Shaded}

\begin{verbatim}
## [1] FALSE
\end{verbatim}

\begin{Shaded}
\begin{Highlighting}[]
\NormalTok{int_example2 <-}\StringTok{ }\KeywordTok{as.integer}\NormalTok{(}\DecValTok{5}\NormalTok{)}
\KeywordTok{is.integer}\NormalTok{(int_example2)}
\end{Highlighting}
\end{Shaded}

\begin{verbatim}
## [1] TRUE
\end{verbatim}

\begin{Shaded}
\begin{Highlighting}[]
\NormalTok{int_example3 <-}\StringTok{ }\KeywordTok{as.integer}\NormalTok{(}\FloatTok{5.9999}\NormalTok{)}
\NormalTok{int_example3}
\end{Highlighting}
\end{Shaded}

\begin{verbatim}
## [1] 5
\end{verbatim}

\textbf{Character}

\begin{itemize}
\tightlist
\item
  Character data represent text, which are called strings. Text data
  that is enclosed in either double or single quotes is interpreted as a
  string.
\item
  example: ``i am a string'', ``ABCD''
\item
  \emph{Test function}: is.character()
\item
  \emph{Coercion function}: as.character()
\end{itemize}

\begin{Shaded}
\begin{Highlighting}[]
\NormalTok{char_example <-}\StringTok{ }\KeywordTok{c}\NormalTok{(}\StringTok{"a"}\NormalTok{, }\StringTok{"b"}\NormalTok{, }\StringTok{"c"}\NormalTok{)}
\KeywordTok{class}\NormalTok{(char_example)}
\end{Highlighting}
\end{Shaded}

\begin{verbatim}
## [1] "character"
\end{verbatim}

\begin{Shaded}
\begin{Highlighting}[]
\NormalTok{num_example}
\end{Highlighting}
\end{Shaded}

\begin{verbatim}
## [1]  4.00 -4.00  4.40 -4.04
\end{verbatim}

\begin{Shaded}
\begin{Highlighting}[]
\KeywordTok{is.character}\NormalTok{(num_example)}
\end{Highlighting}
\end{Shaded}

\begin{verbatim}
## [1] FALSE
\end{verbatim}

\begin{Shaded}
\begin{Highlighting}[]
\NormalTok{char_example2 <-}\StringTok{ }\KeywordTok{as.character}\NormalTok{(num_example)}
\NormalTok{char_example2}
\end{Highlighting}
\end{Shaded}

\begin{verbatim}
## [1] "4"     "-4"    "4.4"   "-4.04"
\end{verbatim}

\begin{Shaded}
\begin{Highlighting}[]
\KeywordTok{is.character}\NormalTok{(char_example2)}
\end{Highlighting}
\end{Shaded}

\begin{verbatim}
## [1] TRUE
\end{verbatim}

\textbf{Logical}

\begin{itemize}
\tightlist
\item
  Logical values represent Boolean values, a binary value having only
  two options. In R, Boolean values are TRUE or FALSE
\item
  \emph{Test function}: is.logical()
\item
  \emph{Coercion function}: as.logical()
\end{itemize}

\begin{Shaded}
\begin{Highlighting}[]
\NormalTok{log_example <-}\StringTok{ }\KeywordTok{c}\NormalTok{(}\OtherTok{TRUE}\NormalTok{, }\OtherTok{FALSE}\NormalTok{)}
\KeywordTok{is.logical}\NormalTok{(log_example)}
\end{Highlighting}
\end{Shaded}

\begin{verbatim}
## [1] TRUE
\end{verbatim}

\begin{Shaded}
\begin{Highlighting}[]
\NormalTok{num_example}
\end{Highlighting}
\end{Shaded}

\begin{verbatim}
## [1]  4.00 -4.00  4.40 -4.04
\end{verbatim}

\begin{Shaded}
\begin{Highlighting}[]
\KeywordTok{is.logical}\NormalTok{(num_example)}
\end{Highlighting}
\end{Shaded}

\begin{verbatim}
## [1] FALSE
\end{verbatim}

\begin{Shaded}
\begin{Highlighting}[]
\NormalTok{log_example2 <-}\StringTok{ }\KeywordTok{as.logical}\NormalTok{(num_example)}
\NormalTok{log_example2}
\end{Highlighting}
\end{Shaded}

\begin{verbatim}
## [1] TRUE TRUE TRUE TRUE
\end{verbatim}

\begin{Shaded}
\begin{Highlighting}[]
\KeywordTok{is.logical}\NormalTok{(log_example2)}
\end{Highlighting}
\end{Shaded}

\begin{verbatim}
## [1] TRUE
\end{verbatim}

\subsection{Special Values}\label{special-values}

There are four special values in R that may cause problems in your
analysis, whether you are aware of it or not. These values are NA, NULL,
Inf/-Inf, and NaN.

\textbf{NA}

\begin{itemize}
\tightlist
\item
  ``Not Available''
\item
  NA represents missing data and any function performed on NA will
  result in NA
\item
  A few ways to handle them:

  \begin{itemize}
  \tightlist
  \item
    is.na()
  \item
    na.omit()
  \item
    complete.cases()
  \item
    na.rm = TRUE (this is an argument for a function)
  \item
    sort(x, na.last = TRUE)
  \end{itemize}
\end{itemize}

\textbf{NULL}

\begin{itemize}
\tightlist
\item
  ``No value, none''
\item
  Null represents not having a value, which is different than the
  missing data of NA
\item
  A few ways to handle them:

  \begin{itemize}
  \tightlist
  \item
    is.null()
  \end{itemize}
\end{itemize}

\textbf{Inf/-Inf}

\begin{itemize}
\tightlist
\item
  ``Positive infinity, negative infinity''
\item
  A few ways to handle them:

  \begin{itemize}
  \tightlist
  \item
    is.infite()
  \item
    is.finite()
  \end{itemize}
\end{itemize}

\textbf{NaN}

\begin{itemize}
\tightlist
\item
  ``Not a Number''
\item
  Values that are not numbers include: 0/0, infinity, negative infinity
\item
  A few ways to handle them:

  \begin{itemize}
  \tightlist
  \item
    is.nan()
  \item
    coerce all NaN values to NA using is.na()
  \end{itemize}
\end{itemize}

\section{Working with Data}\label{working-with-data}

\subsection{Where am I?}\label{where-am-i}

You can find your current working directory by using getwd() and setting
the working directory with setwd(). You will want to replace my path
with the path of your choosing.

Using RStudio, in the Files tab you can navigate to a directory, select
More, and then click Set as Working Directory.

\begin{Shaded}
\begin{Highlighting}[]
\KeywordTok{getwd}\NormalTok{()}
\end{Highlighting}
\end{Shaded}

\begin{verbatim}
## [1] "/Users/stacey/Data/GitHub/Introduction-to-R"
\end{verbatim}

\begin{Shaded}
\begin{Highlighting}[]
\KeywordTok{setwd}\NormalTok{(}\StringTok{"/Users/stacey/Data/GitHub/Introduction-to-R"}\NormalTok{)}
\end{Highlighting}
\end{Shaded}

\subsection{Reading Data into R}\label{reading-data-into-r}

The following uses the combined\_out.txt file that was generated in the
\href{https://sites.google.com/view/bioinformaticssg/presentations/read-counts}{Read
Counts} meeting. Following the steps from the tutorial should generate
the same file I have posted on the
\href{https://github.com/bioinformaticssg/Introduction-to-R}{BioinformaticsSG
GitHub page}. To get this file just follow the link to the GitHub page,
click on the green \textbf{Clone or download} button in the upper right
corner, and choose \textbf{Download ZIP}.

There are two common ways to read data files into R, depending on the
type of file they are. For comma separated value (CSV) files use
read.csv() and for tab-delimited files use read.table(). There are
options for reading in data of other types and code from a script but we
will focus on these today for our data.

For the read.table() command below, replace the path I have provided
with your path to the file. If you have already set the working
directory and it contains your file, you can just provide the file name.

Note: the path and file names are strings and must be enclosed in
quotes.

\begin{Shaded}
\begin{Highlighting}[]
\KeywordTok{read.table}\NormalTok{(}\StringTok{"/Users/stacey/Data/GitHub/Introduction-to-R/combined_out.txt"}\NormalTok{)}
\end{Highlighting}
\end{Shaded}

When you run the read.table() command, you will see the data file
printed to your console. It would be nice if it could be referred to
without having to type out the read.table() command, so let's assign a
variable to the data table.

\begin{Shaded}
\begin{Highlighting}[]
\NormalTok{my_data <-}\StringTok{ }\KeywordTok{read.table}\NormalTok{(}\StringTok{"/Users/stacey/Data/GitHub/Introduction-to-R/combined_out.txt"}\NormalTok{)}
\end{Highlighting}
\end{Shaded}

Now it will be easier to access and use the data table. You can now look
at the top of the data using head(), at the end using tail(), determine
the number of columns using ncol() or length(), the number of rows using
nrow(), and the dimensions of the data dim().

\begin{Shaded}
\begin{Highlighting}[]
\KeywordTok{head}\NormalTok{(my_data)}
\end{Highlighting}
\end{Shaded}

\begin{verbatim}
##                  V1 V2 V3 V4
## 1 ENSG00000277248.1  0  0  0
## 2 ENSG00000274237.1  0  0  0
## 3 ENSG00000280363.1  0  0  0
## 4 ENSG00000279973.1  0  0  0
## 5 ENSG00000226444.2  0  0  0
## 6 ENSG00000276871.1  0  0  0
\end{verbatim}

\begin{Shaded}
\begin{Highlighting}[]
\KeywordTok{tail}\NormalTok{(my_data)}
\end{Highlighting}
\end{Shaded}

\begin{verbatim}
##              V1  V2   V3   V4
## 1458 ERCC-00163   5    7   13
## 1459 ERCC-00164   0    1    0
## 1460 ERCC-00165  33   64   41
## 1461 ERCC-00168   2    1    1
## 1462 ERCC-00170   1    2    3
## 1463 ERCC-00171 916 1253 1056
\end{verbatim}

\begin{Shaded}
\begin{Highlighting}[]
\KeywordTok{ncol}\NormalTok{(my_data)}
\end{Highlighting}
\end{Shaded}

\begin{verbatim}
## [1] 4
\end{verbatim}

\begin{Shaded}
\begin{Highlighting}[]
\KeywordTok{length}\NormalTok{(my_data)}
\end{Highlighting}
\end{Shaded}

\begin{verbatim}
## [1] 4
\end{verbatim}

\begin{Shaded}
\begin{Highlighting}[]
\KeywordTok{nrow}\NormalTok{(my_data)}
\end{Highlighting}
\end{Shaded}

\begin{verbatim}
## [1] 1463
\end{verbatim}

\begin{Shaded}
\begin{Highlighting}[]
\KeywordTok{dim}\NormalTok{(my_data)}
\end{Highlighting}
\end{Shaded}

\begin{verbatim}
## [1] 1463    4
\end{verbatim}

\subsection{Changing Column Names}\label{changing-column-names}

As a default, read.table() assumes the data file does not have a first
row with the names of each column. Although this is true for our data,
it will not always be. If your data does have a header, just set the
header argument to TRUE like so - read.table(``file.txt'', header=TRUE).
You will notice, the header that R made for us when header=FALSE is V1,
V2, V3, V4.

Let's give our column names something more intuitive. First, look at a
list of the column names using colnames().

\begin{Shaded}
\begin{Highlighting}[]
\KeywordTok{colnames}\NormalTok{(my_data)}
\end{Highlighting}
\end{Shaded}

\begin{verbatim}
## [1] "V1" "V2" "V3" "V4"
\end{verbatim}

There are four columns that need new names. I am going to make a vector
of strings that will represent each column's new name. Then I will
assign the column names to be the names in the vector.

\begin{Shaded}
\begin{Highlighting}[]
\KeywordTok{colnames}\NormalTok{(my_data)}
\end{Highlighting}
\end{Shaded}

\begin{verbatim}
## [1] "V1" "V2" "V3" "V4"
\end{verbatim}

\begin{Shaded}
\begin{Highlighting}[]
\NormalTok{new_names <-}\StringTok{ }\KeywordTok{c}\NormalTok{(}\StringTok{"GeneID"}\NormalTok{, }\StringTok{"Sample_1"}\NormalTok{, }\StringTok{"Sample_2"}\NormalTok{, }\StringTok{"Sample_3"}\NormalTok{)}
\KeywordTok{colnames}\NormalTok{(my_data) <-}\StringTok{ }\NormalTok{new_names}
\KeywordTok{colnames}\NormalTok{(my_data)}
\end{Highlighting}
\end{Shaded}

\begin{verbatim}
## [1] "GeneID"   "Sample_1" "Sample_2" "Sample_3"
\end{verbatim}

\begin{Shaded}
\begin{Highlighting}[]
\KeywordTok{head}\NormalTok{(my_data)}
\end{Highlighting}
\end{Shaded}

\begin{verbatim}
##              GeneID Sample_1 Sample_2 Sample_3
## 1 ENSG00000277248.1        0        0        0
## 2 ENSG00000274237.1        0        0        0
## 3 ENSG00000280363.1        0        0        0
## 4 ENSG00000279973.1        0        0        0
## 5 ENSG00000226444.2        0        0        0
## 6 ENSG00000276871.1        0        0        0
\end{verbatim}

\subsection{Indexing and Subsetting}\label{indexing-and-subsetting}

When you read tabular data into R using read.csv() or read.table() it is
stored as a dataframe. Just as any data table, dataframes have rows for
observations and columns for each variable of the dataset. Each column
of a dataframe is a vector and contains the same type of data as
previously described. A dataframe, however, contains vectors of
different types and that is why they exist.

More often then not, we want to select certain columns or rows to work
with, which is possible with an understanding of indexing and
subsetting.

First, let's take a look at indexing. R vectors are 1-indexed, which
means that the first element in a vector is index 1. We can select the
element(s) we want by calling the appropriate index/indices in brackets
(example: vector{[}index{]}). Here \textbf{alphabet} is a vector of
letters that you can select elements from.

\begin{Shaded}
\begin{Highlighting}[]
\NormalTok{alphabet <-}\StringTok{ }\NormalTok{LETTERS}
\NormalTok{alphabet}
\end{Highlighting}
\end{Shaded}

\begin{verbatim}
##  [1] "A" "B" "C" "D" "E" "F" "G" "H" "I" "J" "K" "L" "M" "N" "O" "P" "Q"
## [18] "R" "S" "T" "U" "V" "W" "X" "Y" "Z"
\end{verbatim}

\begin{Shaded}
\begin{Highlighting}[]
\NormalTok{alphabet[}\DecValTok{1}\NormalTok{]}
\end{Highlighting}
\end{Shaded}

\begin{verbatim}
## [1] "A"
\end{verbatim}

\begin{Shaded}
\begin{Highlighting}[]
\NormalTok{alphabet[}\DecValTok{10}\NormalTok{]}
\end{Highlighting}
\end{Shaded}

\begin{verbatim}
## [1] "J"
\end{verbatim}

\begin{Shaded}
\begin{Highlighting}[]
\NormalTok{alphabet[}\DecValTok{1}\OperatorTok{:}\DecValTok{4}\NormalTok{]}
\end{Highlighting}
\end{Shaded}

\begin{verbatim}
## [1] "A" "B" "C" "D"
\end{verbatim}

\begin{Shaded}
\begin{Highlighting}[]
\NormalTok{alphabet[}\KeywordTok{c}\NormalTok{(}\DecValTok{8}\NormalTok{, }\DecValTok{5}\NormalTok{, }\DecValTok{12}\NormalTok{, }\DecValTok{12}\NormalTok{, }\DecValTok{15}\NormalTok{)]}
\end{Highlighting}
\end{Shaded}

\begin{verbatim}
## [1] "H" "E" "L" "L" "O"
\end{verbatim}

You can also remove elements in the same manner.

\begin{Shaded}
\begin{Highlighting}[]
\NormalTok{alphabet[}\OperatorTok{-}\DecValTok{1}\NormalTok{]}
\end{Highlighting}
\end{Shaded}

\begin{verbatim}
##  [1] "B" "C" "D" "E" "F" "G" "H" "I" "J" "K" "L" "M" "N" "O" "P" "Q" "R"
## [18] "S" "T" "U" "V" "W" "X" "Y" "Z"
\end{verbatim}

\begin{Shaded}
\begin{Highlighting}[]
\NormalTok{alphabet[}\OperatorTok{-}\NormalTok{(}\DecValTok{1}\OperatorTok{:}\DecValTok{13}\NormalTok{)]}
\end{Highlighting}
\end{Shaded}

\begin{verbatim}
##  [1] "N" "O" "P" "Q" "R" "S" "T" "U" "V" "W" "X" "Y" "Z"
\end{verbatim}

\begin{Shaded}
\begin{Highlighting}[]
\NormalTok{alphabet[}\OperatorTok{-}\KeywordTok{c}\NormalTok{(}\DecValTok{1}\NormalTok{, }\DecValTok{26}\NormalTok{)]}
\end{Highlighting}
\end{Shaded}

\begin{verbatim}
##  [1] "B" "C" "D" "E" "F" "G" "H" "I" "J" "K" "L" "M" "N" "O" "P" "Q" "R"
## [18] "S" "T" "U" "V" "W" "X" "Y"
\end{verbatim}

You can select elements of a dataframe as well but a little differently.
A common way to access columns of a dataframe is by using the \$
operator and the name of the column. I am not showing the output but you
should she a list printed on the console window.

\begin{Shaded}
\begin{Highlighting}[]
\NormalTok{my_data}\OperatorTok{$}\NormalTok{GeneID}
\NormalTok{my_data}\OperatorTok{$}\NormalTok{Sample_}\DecValTok{1}
\end{Highlighting}
\end{Shaded}

Alternatively, you can use brackets just as we did in the vector
example. Since dataframes have two dimensions (rows and columns) the
bracket operator can take two indexes separated by a comma (e.g. {[}row,
column{]}). Omitting the row index will return all the rows (e.g. {[},
column{]}) and ommitting the column index will return all the columns
(e.g. {[}row, {]}).

\begin{Shaded}
\begin{Highlighting}[]
\NormalTok{my_data[}\DecValTok{1}\NormalTok{, ]}
\end{Highlighting}
\end{Shaded}

\begin{verbatim}
##              GeneID Sample_1 Sample_2 Sample_3
## 1 ENSG00000277248.1        0        0        0
\end{verbatim}

\begin{Shaded}
\begin{Highlighting}[]
\NormalTok{my_data[}\DecValTok{1}\OperatorTok{:}\DecValTok{4}\NormalTok{, ]}
\end{Highlighting}
\end{Shaded}

\begin{verbatim}
##              GeneID Sample_1 Sample_2 Sample_3
## 1 ENSG00000277248.1        0        0        0
## 2 ENSG00000274237.1        0        0        0
## 3 ENSG00000280363.1        0        0        0
## 4 ENSG00000279973.1        0        0        0
\end{verbatim}

\begin{Shaded}
\begin{Highlighting}[]
\NormalTok{my_data[}\DecValTok{1}\OperatorTok{:}\DecValTok{4}\NormalTok{, }\DecValTok{1}\NormalTok{]}
\end{Highlighting}
\end{Shaded}

\begin{verbatim}
## [1] ENSG00000277248.1 ENSG00000274237.1 ENSG00000280363.1 ENSG00000279973.1
## 1463 Levels: ENSG00000008735.13 ENSG00000015475.18 ... ERCC-00171
\end{verbatim}

\begin{Shaded}
\begin{Highlighting}[]
\NormalTok{my_data[}\DecValTok{1}\OperatorTok{:}\DecValTok{4}\NormalTok{, }\DecValTok{1}\OperatorTok{:}\DecValTok{2}\NormalTok{]}
\end{Highlighting}
\end{Shaded}

\begin{verbatim}
##              GeneID Sample_1
## 1 ENSG00000277248.1        0
## 2 ENSG00000274237.1        0
## 3 ENSG00000280363.1        0
## 4 ENSG00000279973.1        0
\end{verbatim}

\begin{Shaded}
\begin{Highlighting}[]
\NormalTok{my_data[}\DecValTok{1}\OperatorTok{:}\DecValTok{4}\NormalTok{, }\KeywordTok{c}\NormalTok{(}\DecValTok{1}\NormalTok{, }\DecValTok{3}\NormalTok{)]}
\end{Highlighting}
\end{Shaded}

\begin{verbatim}
##              GeneID Sample_2
## 1 ENSG00000277248.1        0
## 2 ENSG00000274237.1        0
## 3 ENSG00000280363.1        0
## 4 ENSG00000279973.1        0
\end{verbatim}

\begin{Shaded}
\begin{Highlighting}[]
\NormalTok{my_data[}\DecValTok{1}\OperatorTok{:}\DecValTok{4}\NormalTok{, }\StringTok{"GeneID"}\NormalTok{]}
\end{Highlighting}
\end{Shaded}

\begin{verbatim}
## [1] ENSG00000277248.1 ENSG00000274237.1 ENSG00000280363.1 ENSG00000279973.1
## 1463 Levels: ENSG00000008735.13 ENSG00000015475.18 ... ERCC-00171
\end{verbatim}

\begin{Shaded}
\begin{Highlighting}[]
\NormalTok{my_data[}\DecValTok{1}\OperatorTok{:}\DecValTok{4}\NormalTok{, }\KeywordTok{c}\NormalTok{(}\StringTok{"GeneID"}\NormalTok{, }\StringTok{"Sample_1"}\NormalTok{)]}
\end{Highlighting}
\end{Shaded}

\begin{verbatim}
##              GeneID Sample_1
## 1 ENSG00000277248.1        0
## 2 ENSG00000274237.1        0
## 3 ENSG00000280363.1        0
## 4 ENSG00000279973.1        0
\end{verbatim}

\begin{Shaded}
\begin{Highlighting}[]
\NormalTok{my_data[}\DecValTok{1}\OperatorTok{:}\DecValTok{4}\NormalTok{, }\KeywordTok{c}\NormalTok{(}\StringTok{"GeneID"}\NormalTok{, }\StringTok{"Sample_2"}\NormalTok{)]}
\end{Highlighting}
\end{Shaded}

\begin{verbatim}
##              GeneID Sample_2
## 1 ENSG00000277248.1        0
## 2 ENSG00000274237.1        0
## 3 ENSG00000280363.1        0
## 4 ENSG00000279973.1        0
\end{verbatim}

\subsection{Changing Row Names}\label{changing-row-names}

When you look a the data table, you will notice the left most column are
numbers, these are your row names. Some programs require that your row
names are set to your gene names and expect the gene name column to be
excluded.

We can do this by using row.names() and subsetting the column containing
the gene names. We will then exclude the gene name column and re-assign
the my\_data variable so it will contain the modifications we just
created.

Note: Changing the column names or row names modifies the dataframe and
does not require a re-assignment of the variable that the dataset is
saved as.

\begin{Shaded}
\begin{Highlighting}[]
\KeywordTok{head}\NormalTok{(my_data)}
\end{Highlighting}
\end{Shaded}

\begin{verbatim}
##              GeneID Sample_1 Sample_2 Sample_3
## 1 ENSG00000277248.1        0        0        0
## 2 ENSG00000274237.1        0        0        0
## 3 ENSG00000280363.1        0        0        0
## 4 ENSG00000279973.1        0        0        0
## 5 ENSG00000226444.2        0        0        0
## 6 ENSG00000276871.1        0        0        0
\end{verbatim}

\begin{Shaded}
\begin{Highlighting}[]
\KeywordTok{row.names}\NormalTok{(my_data)[}\DecValTok{1}\OperatorTok{:}\DecValTok{10}\NormalTok{]}
\end{Highlighting}
\end{Shaded}

\begin{verbatim}
##  [1] "1"  "2"  "3"  "4"  "5"  "6"  "7"  "8"  "9"  "10"
\end{verbatim}

\begin{Shaded}
\begin{Highlighting}[]
\KeywordTok{row.names}\NormalTok{(my_data) <-}\StringTok{ }\NormalTok{my_data}\OperatorTok{$}\NormalTok{GeneID}
\KeywordTok{head}\NormalTok{(my_data)}
\end{Highlighting}
\end{Shaded}

\begin{verbatim}
##                              GeneID Sample_1 Sample_2 Sample_3
## ENSG00000277248.1 ENSG00000277248.1        0        0        0
## ENSG00000274237.1 ENSG00000274237.1        0        0        0
## ENSG00000280363.1 ENSG00000280363.1        0        0        0
## ENSG00000279973.1 ENSG00000279973.1        0        0        0
## ENSG00000226444.2 ENSG00000226444.2        0        0        0
## ENSG00000276871.1 ENSG00000276871.1        0        0        0
\end{verbatim}

\begin{Shaded}
\begin{Highlighting}[]
\NormalTok{my_data <-}\StringTok{ }\NormalTok{my_data[, }\OperatorTok{-}\DecValTok{1}\NormalTok{]}
\KeywordTok{head}\NormalTok{(my_data)}
\end{Highlighting}
\end{Shaded}

\begin{verbatim}
##                   Sample_1 Sample_2 Sample_3
## ENSG00000277248.1        0        0        0
## ENSG00000274237.1        0        0        0
## ENSG00000280363.1        0        0        0
## ENSG00000279973.1        0        0        0
## ENSG00000226444.2        0        0        0
## ENSG00000276871.1        0        0        0
\end{verbatim}

\begin{Shaded}
\begin{Highlighting}[]
\KeywordTok{row.names}\NormalTok{(my_data)[}\DecValTok{1}\OperatorTok{:}\DecValTok{10}\NormalTok{]}
\end{Highlighting}
\end{Shaded}

\begin{verbatim}
##  [1] "ENSG00000277248.1" "ENSG00000274237.1" "ENSG00000280363.1"
##  [4] "ENSG00000279973.1" "ENSG00000226444.2" "ENSG00000276871.1"
##  [7] "ENSG00000277683.1" "ENSG00000277821.1" "ENSG00000273663.1"
## [10] "ENSG00000278534.1"
\end{verbatim}

If you want a shortcut to renaming your rows, you can assign the column
you want to be your row names when you first read your data in R. You
know that index 1 is the ``GeneID'' column so you can pass 1 into your
row.names argument.

The column names are untouched and will still need to be modified as
described above.

\begin{Shaded}
\begin{Highlighting}[]
\NormalTok{named_data <-}\StringTok{ }\KeywordTok{read.table}\NormalTok{(}\StringTok{"/Users/stacey/Data/GitHub/Introduction-to-R/combined_out.txt"}\NormalTok{,}
                         \DataTypeTok{row.names=}\DecValTok{1}\NormalTok{)}
\KeywordTok{head}\NormalTok{(named_data)}
\end{Highlighting}
\end{Shaded}

\begin{verbatim}
##                   V2 V3 V4
## ENSG00000277248.1  0  0  0
## ENSG00000274237.1  0  0  0
## ENSG00000280363.1  0  0  0
## ENSG00000279973.1  0  0  0
## ENSG00000226444.2  0  0  0
## ENSG00000276871.1  0  0  0
\end{verbatim}

\begin{Shaded}
\begin{Highlighting}[]
\KeywordTok{rownames}\NormalTok{(named_data)[}\DecValTok{1}\OperatorTok{:}\DecValTok{10}\NormalTok{]}
\end{Highlighting}
\end{Shaded}

\begin{verbatim}
##  [1] "ENSG00000277248.1" "ENSG00000274237.1" "ENSG00000280363.1"
##  [4] "ENSG00000279973.1" "ENSG00000226444.2" "ENSG00000276871.1"
##  [7] "ENSG00000277683.1" "ENSG00000277821.1" "ENSG00000273663.1"
## [10] "ENSG00000278534.1"
\end{verbatim}

\section{Packages}\label{packages}

Packages are collections of R functions, data, and code that help beef
up R's capabilities. It is very likely that someone has written a
package that will help you achieve whatever your ultimate goal may be.
Base R comes with a set of packages but for every new package you want
to add you will have to install it manually -- don't worry, it is easy!

\textbf{CRAN - install.packages()}

\begin{itemize}
\tightlist
\item
  CRAN = Comprehensive R Archive Network
\item
  This option downloads and installs packages from CRAN-like
  repositories or from local files and is the most common way to get
  packages
\item
  Find out more about this using ?install.packages
\end{itemize}

\begin{Shaded}
\begin{Highlighting}[]
\KeywordTok{install.packages}\NormalTok{(}\StringTok{"dplyr"}\NormalTok{)}
\end{Highlighting}
\end{Shaded}

\href{https://bioconductor.org/packages/release/BiocViews.html\#___Software}{\textbf{Bioconductor}}

\begin{itemize}
\tightlist
\item
  R packages for genomic data analysis are usually found on the
  Bioconductor website
\item
  In an upcoming meeting, we will use DESeq2.
  \href{https://bioconductor.org/packages/release/bioc/html/DESeq2.html}{HERE}
  is the Bioconductor page with installation instructions,
  documentation, and other helpful information.
\end{itemize}

\begin{Shaded}
\begin{Highlighting}[]
\KeywordTok{source}\NormalTok{(}\StringTok{"https://bioconductor.org/biocLite.R"}\NormalTok{)}
\KeywordTok{biocLite}\NormalTok{(}\StringTok{"DESeq2"}\NormalTok{)}
\end{Highlighting}
\end{Shaded}

\textbf{RStudio - Packages}

Lastly, RStudio has a feature to help you install packages in a point
and click fashion.

Choose the Packages tab \textgreater{} click Install \textgreater{}
choose Repository(CRAN) \textgreater{} type the package name
\textgreater{} hit Install.

This the same as the install.packages() option so you may need to check
Bioconductor if you don't find the package you are looking for.


\end{document}
